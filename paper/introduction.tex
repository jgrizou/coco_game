%!TEX root = avollmer.tex
\section{Introduction}
In interaction, humans align and effortlessly, maybe even automatically, create common ground in communication \cite{clark1991grounding,pickering2004toward}. For this, they dispose of an immense amount of shared information. They make use of frames and interaction protocols established in the history of interaction. Frames create a common ground about the purpose of the interaction \cite{tomasello2009cultural,rohlfing2013learning} and include ``predictable, recurrent interactive structures'' (\cite{ninio1996pragmatic}, p. 171). Frames thus provide interactants with guidelines about how to behave (a protocol for interaction) and also help interactants to understand the communicative intentions of their interaction partner. Interaction protocols comprise basic behavioral patterns like roles, turns, timing, and exchange mechanisms.
We aim at investigating how these interaction protocols emerge, because it would shed light on the basic mechanisms underlying interaction and inform us about what are the main issues in building robots capable of a similar interactional flexibility to the one humans possess. We are for instance interested in what kind of strategies humans use to align and what kind of meanings of social signals they converge to. Whereas, in the long run, the obtained findings could be used as priors for a robotic system interacting with humans, in an initial step, we first need to conduct research into how interaction protocols are negotiated in human-human interaction.

We designed an experimental setup with which we aim at investigating the processes used by humans to negotiate a protocol of interaction, when they do not already share one. 
In the current paper, we present and justify the method used and mention the very first preliminary results obtained from a pilot study employing the setup.

Humans and robots view the world differently, so if we want to transfer our results to human-robot interaction (HRI), we should not assume that in the interactions we want to investigate, the partners see the world/interaction in the same way. To investigate the process of negotiating an interaction protocol, we thus consider a setup of a joint construction task in which participants assume asymmetric roles: the role of a builder and the role of an architect. With building blocks, the builder should assemble a target structure which is unknown to him/her but which the architect knows. This collaborative construction task with a joint goal renders the communication between participants indispensable and thus the game is not solvable by either one of the participants alone, e.g. with mere exploration. Thus, failing to complete the game successfully is equivalent to failing to communicate successfully.
Communication is not face-to-face but channels are restricted, so that it is not possible for participants to communicate via familiar verbal or non-verbal communication channels, as for example speech or gestures. At the same time, the setup does not constrain all aspects of communication and thereby gives participants much freedom with respect to some features, including timing and rhythm or possible meanings (e.g. of button presses). The setup does not impose a predefined sequence of interaction upon participants, as it is often done in HRI scenarios \cite{akgun12hri}, but still benefits from a laboratory setting in which we do not need to take the full complexity of natural social interaction into account. With the aim to simulate the sending of signals to an interaction partner who does not have the same perceptual capabilities -- similar to an interaction with a robot -- in our study the architect does not know how exactly his/her signals are perceived by the builder. For the successful completion of the thus highly challenging joint task of the game, both participants have to learn how to interact with each other. 

The main contribution of the current paper is the presentation of the novel experimental method of our study. We would like to demonstrate that it allows studying important questions for the understanding of human negotiation of interaction protocols in joint construction tasks and that these questions are very important for HRI in the long-term. Also, we report preliminary results of a first pilot run of the study. We first briefly discuss related work, then present our method and the preliminary results of the pilot study and close with a conclusion in which we suggest first implications of these results for HRI.