%!TEX root = avollmer.tex

\section{Conclusion}

We presented a new experimental method which allows studying important aspects of human communication with high relevance to HRI.
In a first pilot study, we show that two players that never had a chance to interact by the means of a restricted interface before were able to communicate and act upon communicative acts whose meanings were never explicitly negotiated between interaction partners. From our preliminary results, we can already suggest first implications for HRI. These implications are twofold.

(a) Both builder and architect have preconceptions of what interaction frames the other player is likely to understand, trying to use or interpret signals with respect to those frames. The ``feedback frame'' seems the most commonly thought about and the easiest to understand in the context of our experiment.

Humans are capable of solving the kind of communication problem robots can have with humans. We have learned an interesting lesson: humans can solve such restricted asymmetric interaction problems by projecting the interaction into different common frames of interaction and selecting the one that is more coherent with the history of interaction. In this setup such an interaction frame provides information about many properties of the interaction including for example (1) a context (e.g. building something with a limited number of blocks and specific constraints (on a table, flat, \ldots)) and (2) a set of possible meanings (e.g. evaluations, guidance signals, references to colors or shapes), which we acquired from our experience interacting with others. 

Based on such observations, by replacing the builder by an artificial agent (e.g. a robot), we can aim at constructing robots capable of learning a task from human instructions without programming it in advance to understand the human communicative acts and without preprogramming a specific rigid interaction protocol. To do so, we should equip our robot with a set of common interaction frames on which it can rely to find one that is coherent with the interaction history. We have begun to explore this direction using a pick and place experiment where a robot is instructed to reach a specific configuration of objects using raw vocal utterances whose mapping to their associated meaning is unknown at the beginning \cite{grizou2013robot}. We further extended this work to brain-computer interaction (BCI) scenarios enabling calibration-free BCI control of an artificial agent in a reaching task \cite{grizou2014calibration}.

(b) The builder's actions play an important role in the understanding of signals. Meaning is co-constructed by the interaction partners. With his/her propositions of blocks and positions, the builder provides frames in which he/she creates slots for the architect to provide information. And thus the builder's frames constitute the meaning of the architect's input to a large extent. This has also been found in asymmetric and restricted interactions with interaction partners with limited communicational abilities, as for example preverbal infants or impaired persons \cite{ochs1979propositions, goodwin1995co}. A similar mechanism of proposition in a learning robot could be a means to elicit appropriate signals from a human tutor in HRI \cite{cakmak2012designing,vollmer2014robots,cangelosi2010integration}. 
