%!TEX root = avollmer.tex
\section{Related Work}

In this section, we will briefly discuss the most relevant related work, which lie in the field of experimental semiotics, and highlight the novelty of our proposed experimental method.
The field of experimental semiotics studies the emergence and evolution of communication systems \cite{galantucci2009experimental}. Here, instead of computer simulations as conducted by others (see \cite{cangelosi2002simulating,steels2012experiments}), controlled experiments in laboratory settings are designed to observe communication between participants who perform joint tasks. For instance, Galantucci et al. showed that pairs of participants performing a joint task could coordinate their behavior by agreeing on a symbol system \cite{galantucci2005experimental}.

Most experimental semiotics studies developed to study joint action involve symmetric communication (cf. \cite{Galantucci2011experimental}).
Two studies which do consider asymmetric communication are the studies conducted by de Ruiter et al. \cite{de2010exploring} and Griffiths et al. \cite{griffiths2012bottom}. 

In their score- and round-based Tacit Communication Game, de Ruiter et al. investigated the cognitive processes responsible for the development and the recognition of new conventions by looking at reaction time. In a 3-by-3 grid world, two participants each manipulate a shape. For both of the shapes, the ``sender'' sees a target configuration. He/she first has to communicate the other player's target configuration to the other player, the ``receiver'', and second has to bring the own shape to his/her own respective target.
De Ruiter et al. found that participants succeeded $83\%$ of the time and that the timing of movements is used to indicate a position. When comparing success rates for when the sender saw versus did not see the receiver's moves, the authors found that the game involves bidirectional communication and receiving information about the other player facilitates communication. The harder the communicative problem was, the more planning time was needed by both participants.

The setup of the study conducted by Griffiths et al. \cite{griffiths2012bottom} is more directly related to our setup. 
It is based on the alien world game setup by Morlino et al., in which in a square world shown on a computer screen, positions (left or right) and movements (shake horizontally or shake vertically) of 16 objects have to be explored via a mouse to maximize a score \cite{morlino2010developing}. It investigates the learning of categories, so the objects belonged to four categories which were defined by certain properties of the objects. Each category was associated with a target manipulation, i.e. shape and weight determined where an object should be positioned and how it should be moved. In the work by Griffiths et al. the learner could realize this task with the help of information given by a tutor who had prior knowledge about the categories the learner should explore. For this alteration, two players played the originally single player game simultaneously in separate rooms over a network connection. The computer screens in this setup additionally showed six buttons underneath the grid world. The tutor's communication to the learner consisted of the pressing and releasing of these six buttons using a keyboard. This was the only action the tutor could perform on the world.
The authors found that tutors most commonly used yes, no and concrete instructions, such as place and shake, as signals to the learners. Negative feedback was given least often and its amount correlated with task failure. Learners who ignored less signals performed the task better.\par
The main, very important difference between the two asymmetric setups described above and our setup concerns the very nature of the task. Whereas in Griffiths et al.'s study the task is solvable with mere exploration, in our setup the input of the architect is essential. The latter is also the case in the study by de Ruiter et al., but in our setup no score is displayed to either of the players who in our case are not separate learner and tutor, or receiver and sender, but they solve the task together assuming the roles of a builder and an architect. Correspondingly, in our setup, the game does not include multiple episodes or rounds but it is continuous with the builder deciding when the task is completed and the game ends. The game of the study by de Ruiter et al. is based on fixed turns which is not the case with our game, where participants can act simultaneously and react directly upon each others conduct.

By designing a continuous game without displaying a score, interaction remains natural (i.e., free) to a high degree. 

Another important difference which makes our setup novel regards the restriction of communicative channels. In contrast to the other two works, in our setup, the architect is not aware of how his/her actions are presented on the builder side and how they will be perceived. %This renders the situation similar to human-robot interaction.
%This difference should also minimize the use of simple iconic feedback (as for example encoding the manipulation of horizontally shaking the object by alternately pressing one button to the right and one button to the left as reported by Griffiths et al.)
%
% final sentence about summarizing how our setup is novel: what are the benefits of doing it our way?
% - no score
% - no rounds, continuous
% - restricted communication
% - awareness about presentation of actions

