\section{Comments on experimental design}

\begin{itemize}
    \item We can see the hand of the student, which may convey social cues.
    \item The information I decided to record is not enough for a proper analysis, we would need to record the full experiment on both side (user + objects) with participant speaking aloud. Ideally, we want to track, through time, the evolution of belief on both side. The more the time tracking is accurate the more we can count event for each meaning and their evolution.
    \item The colour of the button may be a bias for the teacher. But in an other perspective it helps to remember which button means what. Also, I heard that $7\pm2$ is in average the number of memory association untrained human can store (short/middle term). Then perhaps $10$ buttons is too much. Or more importantly, perhaps the only facts that this experiment is short terms bias it towards the use of simple, easy to remember, instructions.
    \item The construction the teacher asked the student to build may have some specific pattern and properties (symmetry, colour ordering, etc.) that may influence the learning. But those are common assumptions even when learning from known signals.
    \item From my personal experience, there is a little lag between both side (no more than 2 seconds). Streaming the video is not done the best way. It is transparent for the users, the only effect I can think of is that it slows down the interaction and may influence the user towards strong turn taking behaviours. May be good to fix.
    \item In recordings, have an explicit marker to known when experiment start and stop. 
    \item Would be better if we can extract high level information from the current construction without using laborious annotation process or automated image analysis. It may be good to implement the task on a computer, where the learner is using the mouse to build something on the screen. This way we can control what the teacher sees, what the learner can do, and what is the current state of the construction in a usable format.
    \item Would be good to have a measure of progress in the task. How much the task is completed, or how far the learner is going in a good of bad direction. It would enable to correlate signal understanding and task progression, to link reset event to negative performance, and to ``explain'' context relative meanings. If previous point implemented, this may be quite easy. 
    \item What is more relevant: press/release button events or the duration a button is hold pressed? This depends on the situation and is hard to analyse ``automatically''.
    \item How to analyse pattern in the use of one button, e.g.\ one press to mean ``no'', two press to mean ``yes''?
    \item The student reported difficulty to remember the symbols that were used before, perhaps queuing the last five symbols would help. 
    \item Student tends to use the following heuristic: ``the symbol I will receive the most will be a negative feedback''. If the teacher intend to provide it, it is a very good heuristic due to the construction task. Can we think of an other setup where it is no more a good strategy?
\end{itemize}
