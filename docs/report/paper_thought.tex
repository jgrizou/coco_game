\dictentry{Frames}{Social behavioural pattern that guide social interaction by providing predictable, recurrent interactive structures. Embedding a new word within a familiar frame results in the reduction of the information load as this word will be perceived as a new slot within a familiar routine}

\dictentry{Pragmatics}{Subfield of linguistics which studies the ways in which context contributes to meaning. Pragmatics studies how the transmission of meaning depends not only on structural and linguistic knowledge (e.g., grammar, lexicon, etc.) of the speaker and listener, but also on the context of the utterance, any pre-existing knowledge about those involved, the inferred intent of the speaker, and other factors. In this respect, pragmatics explains how language users are able to overcome apparent ambiguity, since meaning relies on the manner, place, time etc.\ of an utterance}

\dictentry{Theory of mind}{Ability to attribute mental states --- beliefs, intents, desires, pretending, knowledge, etc. --- to oneself and others and to understand that others have beliefs, desires, and intentions that are different from one's own. Deficits occur in people with autism spectrum disorders, schizophrenia, attention deficit hyperactivity disorder, as well as neurotoxicity due to alcohol abuse}

\dictentry{Confirmation bias}{Tendency of people to favour information that confirms their beliefs or hypotheses. People display this bias when they gather or remember information selectively, or when they interpret it in a biased way. The effect is stronger for emotionally charged issues and for deeply entrenched beliefs. They also tend to interpret ambiguous evidence as supporting their existing position. Biased search, interpretation and memory have been invoked to explain attitude polarization (when a disagreement becomes more extreme even though the different parties are exposed to the same evidence), belief perseverance (when beliefs persist after the evidence for them is shown to be false), the irrational primacy effect (a greater reliance on information encountered early in a series) and illusory correlation (when people falsely perceive an association between two events or situations)}



\section{Thoughts}

This section summarize few thoughts developed during meetings, mail exchanges and reading of the pragmatic frames paper \cite{rohlfing2013learning}. A small glossary is available at the end of the document.

It is important to note that, in our human experiment and algorithm framework, we are not focused on how to acquired new abilities but rather on how two agents can understand each other in asymmetric interaction. We assume that there is a pre-defined interaction goal or context (building something with blocks) and that both participant knowns how to complete a task inside this context. In our algorithmic framework we assume there is a shared knowledge on the pragmatic frame used in the interaction. While in the human experiment, even if no frames are explicitly given, participants are adults and are already familiar with a multitude of possible frames to apply in such situation. In this case participants needs to agree on which frame(s) their interaction is based on, and which buttons are used to mean what inside this frame.

I would like to explicitly details this point as I think the vocabulary we are learning from this community is very relevant for us. In our human experiment, we could argue that both side are trying to ground the interaction into known frames of interaction. When trying to infer the current mental state of the other, people tend to look for pattern similar to known, and ``universally shared'', interaction frames. Indeed, adults have already acquired complex pragmatics frames which may include feedback, guidance, location or colour naming. The use of the pragmatics is used to identify which signal means what in the context of a frame.
\begin{itemize}
    \item The student uses its knowledge about the possible tasks to ground the signal appearing on the screen to some meaning into a frame. Frame that they believe the teacher may be using with regard to the current state of the interaction.
    \item The teacher uses its observation of the student behaviour to select a particular frame (and press the button accordingly).
\end{itemize}
Some specific signals can also be identified as conveying a certain state of mind. But their interpretation will depend on the current state of the interaction. As an example: a highly unexpected signals, e.g.\ all buttons pressed, may be commonly accepted as a ``stop and pay attention'' signal which according to the context, i.e.\ if the student feel confident or not, may be interpreted as a ``reset'' or a ``finish''.
\\

The first conclusion that as been derived by Katharina is that those experiments are more related to alignment (for more info about alignment see \cite{pickering2004toward}). However, to her knowledge, alignment was not studied in teaching scenarios, i.e.\ in asymmetric interactions, which makes our experiment innovative.

A second perspective is to use this setup for different types of task (construction, classification, \ldots) and study what types of instructions are emerging on each cases. What are the conditions, the basics, of each type of instruction signal. 

A third immediate perspective is on the algorithm side. We could also embed a set of frames inside our robot/agent and try to match to ongoing interaction to one of such frames. That would make hypothesis on both side, tasks and frames. 
\\

We may sometime have some vocabulary misunderstandings due to our respective fields.  The main one is probably on the term ``feedback'' which we tend to use for the specific positive and negative assessment of action while, I think, Katharina is using in a more general term, as an indication from a teacher to a learner.

Additionally, perhaps the word \emph{teacher} and \emph{student} are not adapted to this scenario. It is more like an \emph{architect} and a \emph{worker}, they both know how to build a tower but the architect should explain which tower he wants to be build.
\\

Finally, on an other perspective, I think Katharina would be really interested in building system that can learn (and later use) frames.
