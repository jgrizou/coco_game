\documentclass[a4paper,11pt]{article}
\usepackage[english]{babel} 
\usepackage[hmargin=2cm,vmargin=2.5cm]{geometry} 

\usepackage{caption}
\usepackage{subcaption} 
\usepackage{float}

\usepackage{graphics}
\usepackage{graphicx}

\usepackage{comment} 
\usepackage{paralist}

\usepackage{amssymb}
\usepackage{amsmath}
\usepackage{amsfonts}

\newcommand{\degree}{\ensuremath{^\circ}}

\usepackage{hyperref}

\usepackage{color}
\newcommand{\todo}[1]{{\small\color{red}[#1]}}


\usepackage[nonumberlist]{glossaries}
%\newcommand{\dictentry}[2]{\begin{quotation}\textbf{#1}: #2\end{quotation}}
\newcommand{\dictentry}[2]{%
  \newglossaryentry{#1}{name=#1,description={#2}}%
}
\makeglossaries

\begin{document}

\begin{center}
{\LARGE \textsc{Reference summary} \\ [0.2cm]
{\large \textbf{Last update:} \today}} \\ [0.2cm]
\end{center}

\begin{abstract}
This document is meant to summarize the current readings.
\end{abstract}

\section{Related Work}

\subsection{Experimental Semiotics (ES)}

%--------------------------------------------------------------------------------------------------------------------------------------------------------
\subsubsection{Galantucci}
Galatucci \cite{galantucci2005experimental} uses ES to study the emergence and evolution of communication systems supplementary to computer simulations.

Game 1
\begin{itemize}
\item two-by-two grid, with each one move solve game by being in the same room
\item dyads
\item shared virtual environment
\item no speech, no language
\item round-based
\item common score
\item communication 1 -- 2: drawings on pad, quick fade, constant downward drift
\item communication 2 -- 1: drawings on pad, quick fade, constant downward drift
\item participants practiced communication medium
\item end of game: maximum score
\item measures/graphics: 
\begin{itemize}
\item score plot for each pair/run shows convergence
\item time to solution
\end{itemize}
\item{results}
\begin{itemize}
\item strategies: learning-by-using (opportunistic), naming procedures (convergence much faster)
\item sign systems: numeration based, icon based, map based
\end{itemize}
\end{itemize}

Game 2, 3
\begin{itemize}
\item 2
\begin{itemize}
\item same setup as in 1, same participants that were successful in previous game
\item larger environment
\item prey captured, when both agents in same room
\item continuous, prey appeared in another room upon capture
\end{itemize}
\item 3
\begin{itemize}
\item same setup as in 2, same participants that were successful in previous game
\item larger environment
\item two enemies, one for each agent
\end{itemize}
\item measures/graphics: 
\begin{itemize}
\item score plot for each pair/run shows convergence
\item time to solution
\item minimum score
\end{itemize}
\item{results}
\begin{itemize}
\item 6 out of 8 participant pairs used the same strategy 
\item sign systems: easily diverge, can integrate new mappings with old ones
\item importance of silent behavior-coordination (e.g. split search)
\item parsimony: once sign system has emerged, new signs unrelated to the ones in use were rarely developed.
\end{itemize}
\end{itemize} 
%--------------------------------------------------------------------------------------------------------------------------------------------------------

In \cite{galantucci2011experimental}, Galantucci gives an overview of ES research.

Varieties of studies
\begin{itemize}
\item referential (e.g. draw pieces of music for a partner to identify, Healey 2000, assign pre-established referents to be communicated)
\item coordination (e.g. players freely discover/negotiate shared referents, \cite{galantucci2005experimental})
\item matching (closed set of communication forms, closed set of referents, e.g. communicate about fixed set of geometric figures by combining letters of a fixed set).So our setup after Galantucci was intended as a semiotic matching game, but people generally do not seem to use the buttons to describe/refer to the bricks?
\end{itemize}

Themes
\begin{itemize}
\item early emergence of linguistic structure
\begin{itemize}
\item emergence of combinatoriality (e.g. in drawings: recurrence of basic forms across signs, the faster forms fade, the more combinatoriality)
\item emergence of compositionality (e.g. having one symbol for circle, one for insert, one for position)
\end{itemize}
\item social manipulations
\begin{itemize}
\item symbols (the emergence of symbols is what we also look at. the relationship between the button presses and what they stand for is arbitrary)
\item evolution of systems within/across generations (e.g. computer simulations, Cangelosi and Parisi 2002, but also ES studies)
\end{itemize}
\end{itemize}

%--------------------------------------------------------------------------------------------------------------------------------------------------------

\subsubsection{De Ruiter}
De Ruiter \cite{de2010exploring}'s main argument, he wants to test is that for successful communication, people need a special kind of capacity, the `interactional intelligence'. (sender: recipient design, receiver: intention recognition), conventions ease processing

The Tacit Communication Game

1
\begin{itemize}
\item Focusses on the cognitive processes in individuals, that are responsible for the development and recognition of newly crewted conventions.
\item reaction times and processing load
\item three-by-three grid, both agents operate shape, sender sees goal configuration (position and rotation)
\item goal: sender moves own object in goal configuration and receiver in his/her respective goal configuration
\item communication sender - receiver: moves and rotations
\item communication receiver - sender: moves and rotations
\item planning time for sender, interpretation time for receiver
\item limited time for predesigned turns
\item round-based
\item binary score, correct/incorrect
\item individual training, joint training
\item measures / graphics
\begin{itemize}
\item planning time of sender, receiver and time to achieve success
\end{itemize}
\item results
\begin{itemize}
\item manner (here:timing) used to indicate a position, `'
\item $83\%$ successful
\end{itemize}
\end{itemize}

2 
\begin{itemize}
	\item Does feedback facilitate communication? (Are senders applying fixed strategy, which receivers simply have to decode?)
	\item on half of trials sender sees receivers moves, on the other half s/he does not. Is success rate influenced?
	\item additional measures
	\begin{itemize}
		\item qualitative analysis
	\end{itemize}
	\item results
	\begin{itemize}
		\item comparing success rates: two-way communication is involved. effectiveness of communication increases when senders have access to receiver feedback.
		\item participants need more planning time for hard trials than easy ones.
		\item if a problem is hard, it is hard for both senders and receivers.
	\end{itemize}
\end{itemize}

3
\begin{itemize}
\item communicative (signal the goal configuration of receiver to them, receiver does not know goal) vs. non-communicative (go to receiver's goal and match orientation as closely as possible then go to own goal, receiver knows their goal)
\item additional measures
	\begin{itemize}
		\item mean pause times in senders' moves
	\end{itemize}
\item results
\begin{itemize}
\item additional processing load for sender in communicative than non-communicative trials
\item the harder the communicative problem, the more planning time is needed by both participants (not dependent on number of moves performed per trial).
\end{itemize}
\end{itemize}
%--------------------------------------------------------------------------------------------------------------------------------------------------------

\subsubsection{Griffiths}
Griffiths et al. \cite{griffiths2012bottom} in their setup investigate learning of categories.

\begin{itemize}
\item score
\item tutor's play several trials with changing learners
\item it is possible to solve the game alone by exploration, without a tutor's feedback
\item tutor and learner have the same view on the workspace. Tutor knows how his/her feedback is perceived.
\item learners might already expect yes/no feedback as they are told in the instructions that there are right and wrong ways of manipulating the objects.
\item measures/graphics
\begin{itemize}
\item understand: learner's interpretation $=$ tutor's intended meaning; misinterpret: learner's interpretation $~=$ tutor's intended meaning; ignore: no learner interpretation for tutor's feedback.
\item qualitative description
\item interview
\item number of button events for types of feedback 
\item number of button events understood, misinterpreted and ignored
\end{itemize}
\item results
\begin{itemize}
\item most common feedback: yes, no, concrete instructions (place, shake)
\item learners who ignored less signals, achieved better performance
\item general strategy in tutors to simplify strategy from learner to learner
\item \# concrete instructions $>$ negative feedback $<$ positive feedback
\item negative correlation between negative feedback and task success
\end{itemize}
\end{itemize}

\subsection{Differences with our setup}

\todo{Galantucci: both sides can generate signals. Game 1: score and round-based, Game 2,3: continuous, score. De Ruiter: involves training, turn-based, several trials, binary "score", variant 1: both players see same scene, variant 2: sender does not see sender agent for half of the trials. Griffiths: score and several episodes. asymmetric, but tutor and learner have the same view on the `workspace'. Us: No score, the user decides when to stop. the architect does not know how exactly his/her feedback is perceived.} 

\section{Motivation}

\subsection{Teleological Stance}
One-year old children do not have a ``theory-of-mind'' yet (emerges with 4 years of age), but they can interpret others'�� actions as being goal-directed, identify the most efficient means to reach a goal, and expect actors to perform this most efficient means. 
Gergely and Csibra put forward an explanation for one-year old infants' interpretation of goal-directed actions \cite{gergely2003teleological}. Instead of an innate mentalistic stance, according to which infants attribute desires and beliefs to subjects (meaning e.g. ``they put themselves into the actor's place''��), the authors suggest that these mentalistic capabilities develop later on and they propose an early teleological stance which is an interpretational system only taking into account the reality states: action, goal state, and situational constraints. The teleological stance follows the principle of rational action stating that actors will take the most rational action available to reach a goal given the situational constraints. \todo{JG: Do you understand how the experiment presented in figure 1 can tell the difference between teleological and mentalistic representation? AV: It doesn't. This violation-of-expectation experiment simply demonstrates the infants' ability to understand goal-directed actions. The authors assumption (based on research in developmental psychology) is that infants up to one year of age are not capable to use a theory of mind, yet. JG: Maybe we should read about ``theory-of-mind'' and why it is said to emerge around 4. Do we want to argue from the teleological representation perspective, our users already have good ``theory-of-mind''.}



\bibliographystyle{ieeetr}
\bibliography{ref}

\glsaddall
\printglossary
\end{document}